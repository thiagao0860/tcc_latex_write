\chapter[Conclusão]{Conclusão}

A realização do trabalho se mostrou de grande valia para o desenvolvimento, tanto pessoal do aluno, quanto para as 
necessidades do laboratório, obtendo pleno sucesso no que tangia ao escopo proposto, sendo possível a realização
do controle discreto dos circuitos propostos utilizando os circuitos em \textit{Edge} e em Nuvem utilizando o protocolo MQTT. 
Mostrando-se diretamente como uma poderosa alternativa aos modelos amplamente utilizados no contexto de sistemas 
elétro-hidráulicos e elétro-pneumáticos atualmente, sendo eles tanto a lógica de relés aplicada diretamente, quanto utilizando \ac{CLP}, ou até métodos 
mais complexos; ao passo que é também uma solução de baixo custo e que simplifica em muito a implementação de circuitos
lógicos de forma facilmente escalável. 

\section{Principais Contribuições}

O principal produto do trabalho realizado é a apresentação de um \textit{framework} para realização do controle de uma sistema
 via MQTT de forma que os códigos utilizados no controle de cada ESP32 são modulares e podem ser facilmente substituidos
 para diversos outros tipos de dados a serem repassados a depender do maquinário a ser utilizado. Além disso, o sistema
 apresenta um segundo grande ganho que é a implementação de um sistema com habilidade de lidar com falhas no sistema de 
 Internet sem perder a capacidade de conectar com a Nuvem.

\section{Trabalhos Futuros}

Apesar do sucesso da comunicação em relação ao funcionamento do sistema, foram encontrados comportamentos que não eram diretamente
 esperados durante o planejamento da aplicação, desta forma podem ser elencados como estudos ainda a serem realizados no 
 futuro: o estudo sobre quais os métodos para reduzir a latência da comunicação, embora para este trabalho cujo escopo se
 resume somente a controle discreto de sistemas isso não seja caracterizado ainda como um problema; outro grande problema
 a ser resolvido no futuro é a inclusão de sistemas capazes de perceber erros nos sensores em tempo real, para reduzir 
 erros como mal contato e por fim fica também como uma possibilidade a ser estudada o uso deste tipo de comunicação para 
 controle não discreto. além disso, existe também um grande interesse do \ac{MAPL} no estudo de quais as melhores maneiras 
 de dimensionar o componente responsável pela lógica do sistema, um aplicativo de controle já está sendo estudado; neste
 trabalho, este componente foi implementado de forma bastante simples em \textit{Python}, como apresentado no experimento 2. E,
 por fim, como trabalho futuro, a longo prazo, podemos elencar a criação de fábricas e laboratórios virtuais apresentando-se
 como alternativas importantes em cenários que limitam a ida presencial a estes ambientes como é o caso da pandemia de Covid-19 enfrentada
 recentemente.