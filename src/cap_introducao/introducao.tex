\chapter[Introdução]{Introdução}

Ao estudar a história das civilizações humanas como um todo, um fato que ganha bastante evidência é a ligação próxima 
entre o desenvolvimento de novas e disruptivas técnicas e grandes transformações sociais e ecônomicas fato que pode 
ser obsevado muito claramente, nas três primeiras revoluções industriais com a inclusão, respectivamente, da máquina 
a vapor, energia elétrica e computadores gerando transformações incomensuráveis no \textit{status} social. Atualmente,
vivemos também um momento de grande transformação dirigido principalmente pelo desenvolvimento das comunicações, expondo 
ao mundo um novo nível de informações, o que, como era de se esperar, também está atrelado a uma nova revolução das 
técnicas, a indústria 4.0. O que está em curso nesta revolução é a redução do custo dos microprocessadores o que propicia
o uso dos mesmos de forma distribuída e dispersa, e permite desmontar o paradigma da lógica e processamento centralizado.
A influência da descentralização e dispersão dos microprocessadores no mundo é perceptível em várias áreas: criando
ambientes virtuais, comunicação entre maquinários e serviços, aumento das 
plataformas "\textit{as a service}", entre outras ramificações.

Entre os vários caminhos abertos na quarta revolução industrial, um dos mais expressivos é o da descentralização de sistemas
de grande complexidade, neste sentido o uso de componentes inteligentes capazes de resolver parte de seus desafios em 
\textit{Edge} (ou localmente) associados a uma infraestrutura flexível\footnote{Entende-se aqui como infraestrutura flexível aquela que 
pode ser facilmente ajustada as necessidades momentâneas.} disponibilizada via nuvem tem se mostrado como uma arquitetura 
extremamente poderosa e promissora. Surge a questão referente ao modo de conectar componentes essenciais da manufatura
nesta arquitetura, seus ganhos e observações da aplicação prática.


\section{Motivação}
A motivação para a realização deste estudo se dá por dois grandes vieses, um deles surge pelo aumento do maquinário 
existente no \ac{MAPL} que aumenta a necessidade de um sistema que auxilie no controle e gerenciamento destes componentes
tanto de forma local como remotamente. Além disso, o grande período de afastamento dos laboratórios por conta da 
pandemia enfrentada nos últimos anos fez aumentar em muito os questionamentos sobre a necessidade de muitos processos 
essencialmente presenciais e acelerou o desenvolvimento de várias tecnologias voltadas a realização de atividades via acesso
remoto, contexto no qual este estudo se insere.

\section{Objetivos e Desafios}
O principal objetivo deste trabalho é a construção de um modelo de comunicação, ou \textit{Framework} \ac{MQTT} para aplicações
acadêmico-industriais, que seja de simples
entendimento e possível adaptação para comunicação com outros tipos de dados, do ponto de vista de código; que seja 
modular, isto é, que possa ser adaptado a diferentes tipos de maquinários sem precisar passar por grandes alterações;
que possa ser acessado remotamente, sem que isso torne o sistema dependente de redes externas; e que, por fim, possa ser 
alterado facilmente dada a alta mutabilidade dos sistemas atuais.

Entre os objetivos específicos do \textit{framework} pode-se citar:

\begin{itemize}
    \item Ser acoplável a diferentes maquinários
    \item Deve ser capaz de ser comunicar com a nuvem, mas não pode ser dependente dela
    \item Os códigos utilizados devem ser modulares e de fácil implementação em outros projetos.
\end{itemize}

\section{Organização do Trabalho}
Seguindo os objetivos descritos acima o presente trabalho, passa pela construção de \textit{Framework} a que se impõe grande versatilidade,
desta forma durante sua construção passar-se-á por uma grande revisão de documentos normativos, manuais de fabricantes e 
documentações oficiais; pois somente assim podemos focar em quais são os limites da aplicação de suas funções diferentemente
de um trabalho onde não se sabe sobre o funcionamento e faz-se um teste para saber se funciona. Em segundo momento serão 
apresentados os requisitos que guiaram o desenvolvimento e; por fim; será realizada uma série de testes, como uma prova de
conceito, comprovando a aplicabilidade do \textit{Framework} de forma prática.

