\chapter[Experimentos e Análise dos Resultados]{Experimentos e Análise dos Resultados}
\label{experimentos}

Como já especificado em capítulos anteriores, este trabalho de dispõe a apresentar um Framework de 
 controle para sistemas discretos de forma a utilizar os principais ganhos do protocolo MQTT para propor soluções mais 
 simples do que as amplamente utilizadas em processos industriais para problemas como gerenciamento da rede, custo dos 
 equipamentos utilizados e entre outros. Neste sentido, o principal teste realizado foram a realização de duas provas de
 conceito simulando sistemas amplamente presentes no âmbito industrial para se observar o comportamento do modelo 
 proposto, além disso, a partir das simulações foram encontradas algumas questões para as quais foram propostos testes 
 especiais assim como descrito a seguir.

\section{Método para a Avaliação}
\label{metodo}
Descreva os métodos utilizados para validar a sua hipótese incluindo as medidas de avaliação, conjunto de parâmetros, bases de dados e os trabalhos com os quais a sua proposta será comparada.

\section{Experimentos}
De acordo com o que foi descrito na Seção \ref{metodo}, apresente os resultados dos seus experimentos. A apresentação dos resultados pode ser feita via gráficos ou tabelas. O importante é que haja clareza.


\section{Avaliação dos Resultados}
\label{avaliacao}

A avaliação dos resultados pode ser feita à medida em que os resultados dos experimentos são apresentados, ou em uma seção separada. É importante que você aponte os acertos e as limitações da sua proposta e justifique os resultados obtidos. É fundamental apresentar evidências de que sua hipótese é verdadeira.